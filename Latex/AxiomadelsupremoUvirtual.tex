\documentclass[5pt]{beamer} % en min\'uscula!
 % fuentes de LaTeX
\usetheme{Madrid} 
\usepackage{multicol}
\usepackage[utf8]{inputenc}
\usepackage{latexsym,amsmath,amssymb} % S\'imbolos
\newtheorem{Teorema}{Teorema}
\newtheorem{Ejemplo}{Ejemplo}
\newtheorem{Definicion}{Definici\'on}
\newtheorem{Corolario}{Corolario}
\newtheorem{Prueba}{Prueba}
\newtheorem{Lema}{Lema}
\newtheorem{Proposicion}{Proposici\'{o}n}
\newtheorem{Observacion}{Observaci\'on}
%\usecolortheme{beaver}
%\usepackage{textpos}
\usepackage{scrextend}
\usepackage{hyperref}
\changefontsizes[8pt]{8pt}
\title{Axioma del Supremo }
%\titlegraphic{\includegraphics[width=2cm]{logo}}
\author[USACH]{{ \vspace{.5cm}Departamento de Matem\'atica y C.C.\\ \vspace{.1cm}Universidad de Santiago de Chile}}
%\author[The author]{\includegraphics[height=1cm,width=2cm]{logo}\\The Author}
%Patricio Cerda Loyola\\
%Mery Choque Valdez }
\institute[ Cálculo I]{ \small 
}
\date{\tiny Coordinación de Cálculo I\\ \vspace{.1 cm} Primera versión Abril 2020}
\begin{document} 
\maketitle
\section{Test section one}
\begin{frame}
\frametitle{Nociones previas}

\begin{Definicion}[Nociones previas]
Antes de presentar el axioma del supremo, axioma de los números reales, estudiaremos una serie de definiciones que sirven para acotar conjuntos: cotas superiores e inferiores, máximos y mínimos,
supremos e ínfimos. \\\vspace{.3cm}
\end{Definicion}

\begin{block}{Acotado Superiormente} Un conjunto $A$ es acotado superiormente si existe un real $M$ que es mayor o igual que todos los elementos del conjunto $A$, es decir,
	$$\mbox{Existe}\,\,M\in\mathbb{R}\,\,\mbox{tal que}:\,\,\, x\leq M\,\,\,\mbox{para todo}\,\,x\in A.$$
Al número $M$, lo llamaremos cota superior de $A$.	
	\vspace{.2cm}

\end{block}

Observación: Cualquier otro real mayor que $M$, también será una cota superior de $A$.
\end{frame}


\begin{frame}
	\begin{block}{Acotado Inferiormente} Un conjunto $A$ es acotado interiormente si existe un real $m$ que es menor o igual que todos los elementos del conjunto $A$, es decir,
		$$\mbox{Existe}\,\,m\in\mathbb{R}\,\,\mbox{tal que}:\,\,\, m\leq x\,\,\,\mbox{para todo}\,\,x\in A.$$
		Al número $m$, lo llamaremos cota inferior de $A$.	
		\vspace{.2cm}		
	\end{block}

\begin{exampleblock}{Ejemplos}
\begin{itemize}
	\item El conjunto $\{2,4,5,6,8,11\}$ es acotado superiormente por cualquier número mayor que 11 y es acotado inferiormente por cualquier número menor que 2.
	
	\item El conjunto $\{x\in\mathbb{R};\, x<3\}$ es acotado superiormente por cualquier número mayor o igual a 3 y no es acotado inferiormente.
	
	\item El conjunto $\{x^2+1;\, -1\leq x\leq 1\}$ es acotado superiormente por cualquier numero mayor o igual a 2.
\end{itemize}
\end{exampleblock}

\end{frame}

\begin{frame}
	\begin{block}{Definición} Un conjunto acotado superior e inferiormente, diremos que es un conjunto acotado.
		\end{block}
	
	Observación: Una forma de demostrar que un conjunto $A$ es acotado, es probar que existe un $C>0$ tal que $|x| < C$ para todo $x\in A$.
	
	\begin{block}{Definición (Máximo)}
		Diremos que un conjunto $A$ posee máximo, si posee una cota superior
		que pertenece al conjunto.\end{block}
	
	\begin{block}{Definición (Mínimo)}
	Diremos que un conjunto $A$ posee mínimo, si posee una cota inferior que
	pertenece al conjunto.
\end{block}
\end{frame}
\section{Test section two}
\begin{frame}		
\begin{alertblock}{Obeservación}
\begin{enumerate}
	\item Note que el máximo de un conjunto es el mayor elemento del conjunto
	y que el mínimo de un conjunto es el menor elemento del conjunto.
	
	\item Note también que si el máximo existe, este es único. Lo mismo ocurre con el mínimo.
\end{enumerate}
	\end{alertblock}			

	\begin{exampleblock}{Ejemplos}
		\begin{itemize}
			\item $A=\{2,4,5,6,8,11\}$ tiene como máximo a 11 y como mínimo a 2.
			\item El conjunto $\{x\in\mathbb{R};\, x<3\}$ no posee mínimo ni máximo.
			\item El conjunto $\{x\in\mathbb{R};\, x\leq3\}$ no posee mínimo y tiene como máximo a 3.
			
			\item En el conjunto $\{x^2+1;\, -1\leq x\leq 1\}$ el mínimo es 1 y el máximo es 2
		\end{itemize}
	\end{exampleblock}
	
\end{frame}

\begin{frame}{Supremo e Ínfimo}
\begin{block}{Definición de Supremo}
	Diremos que un conjunto $A$ posee supremo, si existe un real $S$ que satisface
	las siguientes condiciones:
	\begin{itemize}
		\item $S$ es una cota superior de $A$.
		\item Cualquier otra cota superior de $A$ es mayor que $S$.
	\end{itemize}
Observación: Al real $S$, lo llamaremos supremo de $A$ y denotaremos por $sup\,A$.
\end{block}

\begin{block}{Definición de Ínfimo}
	Diremos que un conjunto $A$ posee ínfimo, si existe un real $I$ que satisface
	las siguientes condiciones:
	\begin{itemize}
		\item $I$ es una cota inferior de $A$.
		\item Cualquier otra cota inferior de $A$ es menor que $I$.
	\end{itemize}
Observación: Al real $I$, lo llamaremos ínfimo de $A$ y denotaremos por $inf\,A$.
\end{block}
\end{frame}

\begin{frame}
	\begin{exampleblock}{Ejemplo}
		\begin{itemize}
			\item $A = (-\infty, 2)$ . Tiene como supremo el valor 2, ya que 2 es cota superior del conjunto y
			cualquier otra cota superior de A será mayor que 2. No tiene ínfimo pues no está acotado inferiormente.
			\item  $A = [-10, 5]$. Está acotado superior e inferiormente y tiene a -10 como ínfimo y a 5 como supremo (-10 es mínimo y 5 es máximo).
		\end{itemize}
	\end{exampleblock}
		
		\begin{block}{Propiedades}
		Sean $A$ y $B$ dos conjuntos, definimos $A + B = \{x + y\, :\, x \in A,\,\, y\in B\}$ y $A \cdot B =
		\{x \cdot y \,: \,x \in A,\,\, y \in B\}$, entonces
		\begin{itemize}
			\item $sup(A + B) = sup(A) + sup(B)$.
			\item $sup(A \cdot B) = sup(A) \cdot sup(B)$. Para $A,B \subset [0,\infty)$.
		\end{itemize}
		\end{block}
		\end{frame}

\begin{frame}{Axioma del Supremo}
	\begin{block}{Axioma del Supremo}
			Todo conjunto no vacío y acotado superiormente posee un supremo.
		\end{block}
	
\begin{exampleblock}{Aplicaciones del Axioma de Supremo}
\begin{itemize}
	\item {\bf Parte Entera}: La parte entera de un real $x > 0$, se definirá como el supremo del	conjunto $A = \{n \in \mathbb{N}\, :\, n \leq x\}$. Esto está bien definido pues el conjunto $A$ es acotado superiormente por $x$ y además $0\in A$. Por lo tanto por el axioma del supremo, el conjunto $A$ posee supremo. Este
	supremo será denotado por $[x]$ y se llamará parte entera de $x$.
	
	\item Los números naturales no son acotados superiormente.
	
	\item {\bf Propiedad Arquimediana}: El conjunto $\mathbb{R}$ es arquimediano, es decir, para todo real $x > 0$, existe un natural $n \in \mathbb{N}$, tal que $n \cdot x > 1$.
	
\end{itemize}	
	
\end{exampleblock}	
	
\end{frame}

\begin{frame}
	\begin{block}{Teorema}
Los racionales son densos en los reales. Esto significa que dados dos reales $x, y$
con $x < y$, entonces existe un racional $r$ tal que $x < r < y$.
	\end{block}

%\begin{block}{Definición}
%	contenidos...
%\end{block}

\end{frame}

\end{document}