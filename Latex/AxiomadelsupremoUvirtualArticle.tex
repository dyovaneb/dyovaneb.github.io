\documentclass[5pt]{article} % en min\'uscula!
 % fuentes de LaTeX
\usepackage{multicol}
\usepackage[utf8]{inputenc}
\usepackage{latexsym,amsmath,amssymb} % S\'imbolos
\newtheorem{Teorema}{Teorema}
\newtheorem{Ejemplo}{Ejemplo}
\newtheorem{Definicion}{Definici\'on}
\newtheorem{Corolario}{Corolario}
\newtheorem{Prueba}{Prueba}
\newtheorem{Lema}{Lema}
\newtheorem{Proposicion}{Proposici\'{o}n}
\newtheorem{Observacion}{Observaci\'on}
%\usecolortheme{beaver}
%\usepackage{textpos}
\usepackage{scrextend}
\usepackage{hyperref}
\changefontsizes[8pt]{8pt}
\title{Axioma del Supremo }
%\titlegraphic{\includegraphics[width=2cm]{logo}}
\author{\vspace{.5cm}Departamento de Matem\'atica y C.C.\\ \vspace{.1cm}Universidad de Santiago de Chile}
%\author[The author]{\includegraphics[height=1cm,width=2cm]{logo}\\The Author}
%Patricio Cerda Loyola\\
%Mery Choque Valdez }
\date{\tiny Coordinacián de Cálculo I\\ \vspace{.1 cm} Primera versián Abril 2020}
\begin{document}
	\maketitle

\section{Nociones previas}
Antes de presentar el axioma del supremo, axioma de los námeros reales, estudiaremos una serie de definiciones que sirven para acotar conjuntos: cotas superiores e inferiores, máximos y mánimos,
supremos e ánfimos. \\\vspace{.3cm}


{Acotado Superiormente} Un conjunto $A$ es acotado superiormente si existe un real $M$ que es mayor o igual que todos los elementos del conjunto $A$, es decir,
	\begin{equation}
        \label{AcotadoSup}
	    \mbox{Existe}\,\,M\in\mathbb{R}\,\,\mbox{tal que}:\,\,\, x\leq M\,\,\,\mbox{para todo}\,\,x\in A.
	\end{equation}
Al námero $M$, lo llamaremos cota superior de $A$.
	\vspace{.2cm}



Observacián: Cualquier otro real mayor que $M$, tambián será una cota superior de $A$.




	{Acotado Inferiormente} Un conjunto $A$ es acotado interiormente si existe un real $m$ que es menor o igual que todos los elementos del conjunto $A$, es decir,
		$$\mbox{Existe}\,\,m\in\mathbb{R}\,\,\mbox{tal que}:\,\,\, m\leq x\,\,\,\mbox{para todo}\,\,x\in A.$$
		Al námero $m$, lo llamaremos cota inferior de $A$.
		\vspace{.2cm}


{Ejemplos}
\begin{itemize}
	\item El conjunto $\{2,4,5,6,8,11\}$ es acotado superiormente por cualquier námero mayor que 11 y es acotado inferiormente por cualquier námero menor que 2.

	\item El conjunto $\{x\in\mathbb{R};\, x<3\}$ es acotado superiormente por cualquier námero mayor o igual a 3 y no es acotado inferiormente.

	\item El conjunto $\{x^2+1;\, -1\leq x\leq 1\}$ es acotado superiormente por cualquier numero mayor o igual a 2.
\end{itemize}





	{Definicián} Un conjunto acotado superior e inferiormente, diremos que es un conjunto acotado.


	Observacián: Una forma de demostrar que un conjunto $A$ es acotado, es probar que existe un $C>0$ tal que $|x| < C$ para todo $x\in A$.

	{Definicián (Máximo)}
		Diremos que un conjunto $A$ posee máximo, si posee una cota superior
		que pertenece al conjunto.

	{Definicián (Mánimo)}
	Diremos que un conjunto $A$ posee mánimo, si posee una cota inferior que
	pertenece al conjunto.




{Obeservacián}
\begin{enumerate}
	\item Note que el máximo de un conjunto es el mayor elemento del conjunto
	y que el mánimo de un conjunto es el menor elemento del conjunto.

	\item Note tambián que si el máximo existe, este es ánico. Lo mismo ocurre con el mánimo.
\end{enumerate}
	

	{Ejemplos}
		\begin{itemize}
			\item $A=\{2,4,5,6,8,11\}$ tiene como máximo a 11 y como mánimo a 2.
			\item El conjunto $\{x\in\mathbb{R};\, x<3\}$ no posee mánimo ni máximo.
			\item El conjunto $\{x\in\mathbb{R};\, x\leq3\}$ no posee mánimo y tiene como máximo a 3.

			\item En el conjunto $\{x^2+1;\, -1\leq x\leq 1\}$ el mánimo es 1 y el máximo es 2
		\end{itemize}
	



\section{Supremo e ánfimo}
{Definicián de Supremo}
	Diremos que un conjunto $A$ posee supremo \eqref{AcotadoSup}, si existe un real $S$ que satisface
	las siguientes condiciones:
	\begin{itemize}
		\item $S$ es una cota superior de $A$.
		\item Cualquier otra cota superior de $A$ es mayor que $S$.
	\end{itemize}
Observacián: Al real $S$, lo llamaremos supremo de $A$ y denotaremos por $sup\,A$.


{Definicián de ánfimo}
	Diremos que un conjunto $A$ posee ánfimo, si existe un real $I$ que satisface
	las siguientes condiciones:
	\begin{itemize}
		\item $I$ es una cota inferior de $A$.
		\item Cualquier otra cota inferior de $A$ es menor que $I$.
	\end{itemize}
Observacián: Al real $I$, lo llamaremos ánfimo de $A$ y denotaremos por $inf\,A$.




	{Ejemplo}
		\begin{itemize}
			\item $A = (-\infty, 2)$ . Tiene como supremo el valor 2, ya que 2 es cota superior del conjunto y
			cualquier otra cota superior de A será mayor que 2. No tiene ánfimo pues no está acotado inferiormente.
			\item  $A = [-10, 5]$. Está acotado superior e inferiormente y tiene a -10 como ánfimo y a 5 como supremo (-10 es mánimo y 5 es máximo).
		\end{itemize}
	

		{Propiedades}
		Sean $A$ y $B$ dos conjuntos, definimos $A + B = \{x + y\, :\, x \in A,\,\, y\in B\}$ y $A \cdot B =
		\{x \cdot y \,: \,x \in A,\,\, y \in B\}$, entonces
		\begin{itemize}
			\item $sup(A + B) = sup(A) + sup(B)$.
			\item $sup(A \cdot B) = sup(A) \cdot sup(B)$. Para $A,B \subset [0,\infty)$.
		\end{itemize}



\section{Axioma del Supremo}
	{Axioma del Supremo}
			Todo conjunto no vacáo y acotado superiormente posee un supremo.


{Aplicaciones del Axioma de Supremo}
\begin{itemize}
	\item {\bf Parte Entera}: La parte entera de un real $x > 0$, se definirá como el supremo del	conjunto $A = \{n \in \mathbb{N}\, :\, n \leq x\}$. Esto está bien definido pues el conjunto $A$ es acotado superiormente por $x$ y además $0\in A$. Por lo tanto por el axioma del supremo, el conjunto $A$ posee supremo. Este
	supremo será denotado por $[x]$ y se llamará parte entera de $x$.

	\item Los námeros naturales no son acotados superiormente.

	\item {\bf Propiedad Arquimediana}: El conjunto $\mathbb{R}$ es arquimediano, es decir, para todo real $x > 0$, existe un natural $n \in \mathbb{N}$, tal que $n \cdot x > 1$.

\end{itemize}






	{Teorema}
Los racionales son densos en los reales. Esto significa que dados dos reales $x, y$
con $x < y$, entonces existe un racional $r$ tal que $x < r < y$.


\end{document}
